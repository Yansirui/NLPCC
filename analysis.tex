\section{Analysis}
\label{sec:anl}
\subsection{Ablation Study}
To evaluate the effectiveness of the two modules in the proposed method—the contrastive learning module and the word classification module, we conducted an ablation study. The experimental results are shown in Table~\ref{cls_tab}. From the experimental results, it can be seen that removing the contrastive learning module leads to an average F1 score decrease of 0.37\%, and removing the word classification module results in an average F1 score decrease of 0.34\%. These findings indicate that the removal of either module results in a decline in the model's performance.

\begin{table}
	\caption{ Ablation study for several methods evaluated on the sentence classification tasks. The evaluation metric is F1 score(\%). CL: Contrastive Learning Module. WC: Word Classification Module.}\label{cls_tab}
	\begin{center}
	\begin{tabular}{lccclc}
		\bottomrule
		Model      & ASCC         & CAIL         & CHIP-CTC     & \multicolumn{1}{c}{FSCC} & Avg.                      \\ \hline
		W2CL-BERT   & ~~~94.93~~~        & ~~~66.83~~~        & ~~~84.06~~~        & ~~~86.08~~~                    & ~~~82.98~~~ \\ \hline
		\multicolumn{1}{c}{w/o CL}     & 94.55(\textcolor{red}{-0.38}) & 66.53(\textcolor{red}{-0.30}) & 83.25(\textcolor{red}{-0.81}) & 85.96(\textcolor{red}{-0.12})             & 82.57(\textcolor{red}{-0.41})              \\ \hline
		\multicolumn{1}{c}{w/o WC}     & 94.39(\textcolor{red}{-0.54}) & 66.76(\textcolor{red}{-0.07}) & 83.28(\textcolor{red}{-0.78}) & 85.87(\textcolor{red}{-0.21})             & 82.58(\textcolor{red}{-0.40})              \\ \hline
		W2CL-RoBERTa & ~~~94.84~~~        & ~~~67.71~~~        & ~~~83.98~~~        & ~~~86.53~~~                    & ~~~83.27~~~ \\ \hline
		\multicolumn{1}{c}{w/o CL}     & 94.70(\textcolor{red}{-0.14})  & 67.33(\textcolor{red}{-0.38}) & 83.35(\textcolor{red}{-0.63}) & 86.38(\textcolor{red}{-0.15})             & 82.94(\textcolor{red}{-0.33})              \\ \hline
		\multicolumn{1}{c}{w/o WC}     & 94.55(\textcolor{red}{-0.29}) & 67.56(\textcolor{red}{-0.15}) & 83.48(\textcolor{red}{-0.50}) & 86.39(\textcolor{red}{-0.14})             & 83.00(\textcolor{red}{-0.27})              \\ \bottomrule
	\end{tabular}
	\end{center}
\end{table}

\subsection{Influence of Batch Size}
During the training phase, the impact of batch size on the model's final performance is illustrated in Table~\ref{bs_tab}. As shown in the table, the model performs best when the batch size is set to 16. For the proposed W2CL method, a larger batch size not only increases the sample diversity within each batch during training but also provides more negative samples for contrastive learning. Additionally, some studies\cite{esimcse,bge} have demonstrated that the larger batch size enhances the effectiveness of contrastive learning.

\begin{table}
	\caption{The model's performance under different batch size}\label{bs_tab}
	\begin{center}
		\begin{tabular}{lccclc}
			\bottomrule
			Model      & ASCC         & CAIL(Wait)         & CHIP-CTC     & \multicolumn{1}{c}{FSCC} & Avg.                      \\ \hline
			W2CL-BERT   & ~~~94.93~~~        & ~~~66.83~~~        & ~~~84.06~~~        & ~~~86.08~~~                    & ~~~82.98~~~ \\ \hline
			\multicolumn{1}{c}{bs=4}     & 93.89(\textcolor{red}{-1.04}) & 66.29(\textcolor{red}{-0.54}) & 83.15(\textcolor{red}{-0.91}) & 85.31(\textcolor{red}{-0.77})             & 82.16(\textcolor{red}{-0.82})              \\ \hline
			\multicolumn{1}{c}{bs=8}     & 94.42(\textcolor{red}{-0.51}) & 66.45(\textcolor{red}{-0.38}) & 83.42(\textcolor{red}{-0.64}) & 85.83(\textcolor{red}{-0.25})             & 82.53(\textcolor{red}{-0.45})              \\ \hline
			W2CL-RoBERTa & ~~~94.84~~~        & ~~~67.71~~~        & ~~~83.98~~~        & ~~~86.53~~~                    & ~~~83.27~~~ \\ \hline
			\multicolumn{1}{c}{bs=4}     & 94.10(\textcolor{red}{-0.74})  & 67.01(\textcolor{red}{-0.70}) & 83.35(\textcolor{red}{-0.63}) & 85.60(\textcolor{red}{-0.93})             & 82.52(\textcolor{red}{-0.75})              \\ \hline
			\multicolumn{1}{c}{bs=8}     & 94.55(\textcolor{red}{-0.29}) & 67.26(\textcolor{red}{-0.45}) & 83.56(\textcolor{red}{-0.42}) & 86.13(\textcolor{red}{-0.40})             & 82.88(\textcolor{red}{-0.39})              \\ \bottomrule
		\end{tabular}
	\end{center}
\end{table}

%\subsection{Resource Consumption}
%The proposed W2CL method designs training tasks at the word level, in contrast to SimCSE and SSCL, which design training tasks at the sentence level, this approach consumes fewer computational resources. As shown in Fig 2, the W2CL method requires less training time compared to SimCSE and SSCL and also utilizes fewer GPU resources.